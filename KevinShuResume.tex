% LaTeX file for resume 
% This file uses the resume document class (res.cls)

\documentclass[margin]{res} 
\usepackage{hyperref}
% the margin option causes section titles to appear to the left of body text 
\textwidth=5.2in % increase textwidth to get smaller right margin
%\usepackage{helvetica} % uses helvetica postscript font (download helvetica.sty)
%\usepackage{newcent}   % uses new century schoolbook postscript font 

\begin{document} 
 
\name{Kevin Shu\\[12pt]\href{https://kevinshu.me}{kevinshu.me}} % the \\[12pt] adds a blank line after name

 
\begin{resume} 
\section{Current Employment}
Postdoctoral Resarcher - California Institute of Technology \\August 2025 - Present
 \begin{itemize} \itemsep -2pt  % reduce space between items
 \item In the department of Computational and Mathematical Sciences.
 \item Supervised by Venkat Chandrasekaran.
 \end{itemize}

\section{Education} 
 PhD in Algorithms, Combinatorics, and Optimization, Georgia Institute of Technology, Atlanta, GA\\ August 2019-May 2024
 \begin{itemize} \itemsep -2pt  % reduce space between items
 \item Under the Mathematics department.
 \item Advised by Grigoriy Blekherman.
 \item 4.0 GPA
 \end{itemize}

B.S. in Mathematics, Computer Science, California Institute of Technology, Pasadena, CA, June 2018 \\
GPA 3.9

\section{Publications}
{\bf  Hidden convexity, optimization, and algorithms on rotation matrices  } with Akshay Ramachandran, and Alex L. Wang\\ 2023
\begin{itemize} \itemsep -2pt %reduce space between items
\item Gives geometric results about projections of the set of orthogonal matrices of a given size.
\item Applicable to problems in sattelite attitude adjustment.
\item Hosted on the \href{https://arxiv.org/search/?query=kevin+shu&searchtype=all&source=header}{arxiv}.
\end{itemize}
{\bf  Quadratic Programming with Sparsity Constraints via Polynomial Roots } \\ 2022
\begin{itemize} \itemsep -2pt %reduce space between items
\item Introduces a novel method for solving sparse quadratic optimization problems, inspired by classical equations from linear algebra involving the determinant.
\item Applicable to data science problems such as sparse linear regression and sparse principal components analysis.
\item Gives a fast, practical algorithm for producing good solutions to such sparse quadratic optimization problems.
\item Hosted on the \href{https://arxiv.org/abs/2208.11143}{arxiv}.
\end{itemize}
{\bf  Linear Principal Minor Polynomials: Hyperbolic Determinantal Inequalities and Spectral Containment } \\ 2022
\begin{itemize} \itemsep -2pt %reduce space between items
\item Introduces a class of polynomials, which generalize the determinant, and generalizes various theorems from classical linear algebra to a `sparse' setting.
\item Published in \textbf{International Mathematics Research Notices} in 2022.
\end{itemize}
{\bf  Approximate PSD-Completion for Generalized Chordal Graphs  } \\ 2021
\begin{itemize} \itemsep -2pt %reduce space between items
\item Follow up to previous paper.
\item Gives quantitative bounds for a certain approximation of a sparse semidefinite program.
\item Hosted on the \href{https://arxiv.org/abs/2107.11436}{arxiv}.
\end{itemize}
{\bf    Extreme Nonnegative Quadratics over Stanley Reisner Varieties} \\ 2021
\begin{itemize} \itemsep -2pt %reduce space between items
\item Uses powerful mathematical concepts from algebraic topology and algebraic geometry to study PSD matrix completion.
\item Useful for detailed understanding of sparse semidefinite programming.
\item Hosted on the \href{https://arxiv.org/abs/2106.13894}{arxiv}.
\end{itemize}
{\bf    Causal Channels} \\ 2021
\begin{itemize} \itemsep -2pt %reduce space between items
\item Describes a connection between causal theory, which is a topic in probability regarding the distinction between correlational and causal data.
\item Gives quantitative methods for analysing possible discrepancies in data due to confounding variables.
\item Hosted on the \href{https://arxiv.org/abs/2103.02834}{arxiv}.
\end{itemize}
{\bf    Hyperbolic Relaxation of k-Locally Positive Semidefinite Matrices,} with Grigoriy Blekherman, Santanu Dey, Shengding Sun\\ 2021
\begin{itemize} \itemsep -2pt %reduce space between items
\item Describes a relaxation of a collection of matrices where certain block submatrices are PSD, giving exact bounds on their minimum eigenvalues.
\item Published in the \textbf{SIAM Journal on Optimization} in 2022.
\end{itemize}
{\bf   Sums of Squares and Sparse Semidefinite Programming,} with Grigoriy Blekherman \\ 2021
\begin{itemize} \itemsep -2pt %reduce space between items
\item Gives approximations of sparse semidefinite programs using ideas from algebraic geometry and graph theory.
\item Published in \textbf{SIAM Journal for Applied Algebra and Geometry} in 2021.
\end{itemize}
{\bf  Syntactic Structures and Code Parameters,} with Matilde Marcolli \\ 2017
\begin{itemize} \itemsep -2pt %reduce space between items
\item Published in \textbf{Mathematics in Computer Science}.
\item Describes the connection between the syntactic parameters framework in linguistics with coding theory and theoretical physics.
\end{itemize}


\section{Talks and Presentations Given}

{\bf  Hyperbolicity Cones and Sparse Optimization,} presented at the MIT LIDS seminar \\ 2023 
\begin{itemize} \itemsep -2pt %reduce space between items
\item Talk on sparse quadratic programming.
\end{itemize}

{\bf  Symmetrically Hyperbolic Polynomials,} presented at the \"Oberwolfach Meeting on New Directions in Algebraic Geometry \\ 2023 
\begin{itemize} \itemsep -2pt %reduce space between items
\item Talk on symmetric hyperbolic polynomials.
\end{itemize}

{\bf  Sparse Quadratic Programs via Polynomial Roots,} presented at the Carnegie Mellon University ACO seminar \\ 2023 
\begin{itemize} \itemsep -2pt %reduce space between items
\item Talk on sparse quadratic programming.
\end{itemize}

{\bf  Sparse Quadratic Programs via Polynomial Roots,} presented at the Centrum Wiskunde and Informatica Networks and Optimization seminar \\ 2022 
\begin{itemize} \itemsep -2pt %reduce space between items
\item Talk on sparse quadratic programming.
\end{itemize}

{\bf  Approximating Sparse Semidefinite Programs,} presented at the INFORMS conference \\ 2021 
\begin{itemize} \itemsep -2pt %reduce space between items
\item Invited talk on sparsity in semidefinite programming.
\end{itemize}


{\bf  Poster on Sparse Semidefinite Programs,} presented at the MIP and IPCO conferences \\ 2021 
\begin{itemize} \itemsep -2pt %reduce space between items
\item Describes work on approximating sparse semidefinite programs using a relaxation based on the relationship between nonnegative and sum-of-squares quadratics.
\item Won honorable mentions in both poster competitions.
\end{itemize}

{\bf  Causal Inference and Optimization,} presented at the ACO Student Seminar \\ 2021
\begin{itemize} \itemsep -2pt %reduce space between items
\item Describes work on connections between causality theory and polynomial optimization.
\end{itemize}

{\bf  Lightning Talk on Hyperbolic Relaxations of Locally-PSD Matrices.
,} 
presented at the ICERM - Symmetry, Randomness, and Computations in Real Algebraic Geometry.
 \\ 2020
\begin{itemize} \itemsep -2pt %reduce space between items
\item Describes work on a relaxation of the positive semidefinite cone where only certain submatrices of a given matrix are required to be positive semidefinite.
\end{itemize}



\section{Academic \\ Honors} 
2022 ACO-ARC Fellowship\\
2022 ARCS Foundation award\\
2021 Honorable Mention at the MIP Conference Poster Competition\\
2021 Honorable Mention at the IPCO Conference Poster Competition\\
2021 David L. Brown Fellowship from the Georgia Tech Math Department\\
2018 National Science Foundation Graduate Research Fellowship Recipient\\
2018 Georgia Institute of Technology President's Fellowship Recipient

\section{Research Experience}
{\bf Visiting Scholar,} Max-Planck Institute for Mathematics in the Sciences, Leipzig, Germany \\ Summer 2022
\begin{itemize} \itemsep -2pt %reduce space between items
\item Working under the supervision of Rainer Sinn and Bernd Sturmfels.
\end{itemize}
{\bf Research Assistantship,} Georgia Tech, Atlanta, GA \\ Summer 2020
\begin{itemize} \itemsep -2pt %reduce space between items
\item Funded in part by NSF grant DMS-1901950 and the ACO department.
\item Advised by Grigoriy Blekherman.
\end{itemize}
 
{\bf Research in Causality,} Caltech, Pasadena, CA \\ Aug 2018-Jul 2019
\begin{itemize} \itemsep -2pt %reduce space between items
\item Developed theory of causality and developed bounds on probability distributions.
\item Applied geometric and linear programming in search of bounds.
\item Advised by Leonard Schulman.
\end{itemize}
 
{\bf Research in Linguistics,} Caltech, Pasadena, CA \\ Aug 2017-Jul 2018
\begin{itemize} \itemsep -2pt %reduce space between items
\item Explored connections between linguistics and physical models.
\item Used algebraic geometry to compute phylogenetic trees.
\item Advised by Matilde Marcolli.
\end{itemize}

\section{Teaching Experience}
{\bf Differential Equations Teaching Assistant,} Georgia Tech, Atlanta, GA \\ Aug 2022-Dec 2022
\begin{itemize} \itemsep -2pt %reduce space between items
    \item Graded homework and gave recitations in second year differential equations course.
\end{itemize}
{\bf Differential Equations Teaching Assistant,} Georgia Tech, Atlanta, GA \\ Aug 2021-Dec 2021
\begin{itemize} \itemsep -2pt %reduce space between items
    \item Graded homework and gave recitations in second year differential equations course.
\end{itemize}
{\bf Number Theory Lecture Assistant,} Georgia Tech, Atlanta, GA \\ Jan 2021-May 2021
\begin{itemize} \itemsep -2pt %reduce space between items
    \item Graded homework and gave office hours for an undergraduate number theory course.
\end{itemize}
{\bf Differential Equations Teaching Assistant,} Georgia Tech, Atlanta, GA \\ Jan 2020-May 2020
\begin{itemize} \itemsep -2pt %reduce space between items
    \item Graded homework and gave recitations in second year differential equations course.
\end{itemize}

{\bf Linear Algebra Teaching Assistant,} Georgia Tech, Atlanta, GA \\ Aug 2019-Dec 2019
\begin{itemize} \itemsep -2pt %reduce space between items
    \item Graded homework and gave recitations in first year linear algebra course.
    \item Covered introductory linear algebra material.
\end{itemize}
 
{\bf Advanced Algorithms Teaching Assistant,} Caltech, Pasadena, CA \\ Jan 2018-Mar 2018
\begin{itemize} \itemsep -2pt %reduce space between items
\item Graded homework and gave recitations in graduate level algorithms course.
\item Covered multiplicative weights learning, streaming algorithms,
spectral graph theory, semidefinite programming.
\end{itemize}
 
{\bf Linear Algebra Teaching Assistant,} Caltech, Pasadena, CA \\ Sep 2017-Dec 2017
\begin{itemize} \itemsep -2pt %reduce space between items
\item Graded homework for the applied linear algebra course, including topics like linear algebra, including spectral graph theory, polynomial
interpolation, and principal components analysis.
\item Gave 1 hour of office hours per week.
\end{itemize}
 
{\bf Introduction to Algorithms Teaching Assistant,} Caltech, Pasadena, CA \\ Jan 2017-Mar 2017
\begin{itemize} \itemsep -2pt %reduce space between items
\item Graded homework and gave recitations in undergraduate level algorithms course.
\item Covered graph algorithms, greedy algorithms, dynamic programming, matroids.
\end{itemize}

\section{Other   Activities} 
       {\bf Organizer for the SIAM AG23 Special Session on Convexity,} Georgia Tech, GA    \\       July 2023
        \begin{itemize} \itemsep -2pt
          \item Part of the SIAM AG23 conference on algebraic geometry.
          \item Organized talks about algebraic methods in convex geometry and optimization.
          \item Included 10 talks from both professors and students.
		 \end{itemize}

       {\bf Organizer for the AMS Special Session on Algebraic Methods in Algorithms,} Georgia Tech, GA    \\       March 2023
        \begin{itemize} \itemsep -2pt
          \item Part of the Spring 2023 Sectional Meeting of the AMS.
          \item Organized talks in the intersection of algebra and algorithms, involving 10 talks from both professors and students.
		 \end{itemize}

       {\bf Representative for the Diversity, Equity, and Inclusion committee,} Georgia Tech, GA    \\         2022-2023
        \begin{itemize} \itemsep -2pt
          \item Organized workshops on inclusive language.
          \item Organized Math Night, a monthly tutoring session for undergraduates with the aim or promoting community.
		 \end{itemize}

       {\bf Senior Class President,} Caltech, Pasadena, CA    \\         2018-2019 
        \begin{itemize} \itemsep -2pt
          \item Organized 200 seniors for events such as class trips and senior gifts.
          \item Implemented a new resource scheduling system based on algorithmic techniques that made a process that used to take several days take only a couple hours.
          \item Organized rooms, dining, and transportation for over a hundred students for the class trip, including finding funding from a variety of sources in order to ensure the trip was possible.
          \item Acted as a liason between faculty and students.
		 \end{itemize}

		{\bf Board of Control Secretary,} Caltech \\   2017
        \begin{itemize} \itemsep -2pt
             \item Handled issues relating to academic dishonesty across the school.
             \item Investigated cases of academic dishonesty .
             \item Determined, in 20 cases of academic dishonesty, what response was required by the school, in some cases involving student suspensions.
             \item Organized meetings, gathered evidence, and acted as liason between the faculty and students.
		 \end{itemize}

\section{Outreach   Activities} 
       {\bf First Year Mentor,} Georgia Tech, Atlanta, Georgia     \\         2020-2021 
        \begin{itemize} \itemsep -2pt
            \item Aided 2 first year graduate students in adjusting to graduate school during pandemic.
          \item Gave advice about course scheduling, time managagement, which helped students succeed in courses.
		 \end{itemize}

       {\bf Directed Reading Program Mentor,} Georgia Tech, Atlanta, Georgia     \\         2020-2021 
        \begin{itemize} \itemsep -2pt
            \item Read through lecture notes on game theory with an undergraduate student
          \item Supervised project involving studying games of imperfect information with chance.
		 \end{itemize}

 \section{Work\\ Experience}
 {\bf Full-time Software Engineer,} Google, Mountain View, CA \\ August 2018-July 2019
 \begin{itemize} \itemsep -2pt  % reduce space between items
 \item Full stack web development for a data labelling service (Crowd-Compute)
 \item Lead an initiative to update authentication/authorization to more modern technologies.
 \item Added a major feature for tracking work in the system.
 \item Managed production releases and infrastructure issues.
 \item Worked with C++, Java.
 \end{itemize}
 
{\bf Software Engineering Intern,} Google, Mountain View, CA \\ Aug 2018-Jul 2019
\begin{itemize} \itemsep -2pt %reduce space between items
\item Gathered data from online sources by parsing Reddit pages.
\item Built a machine learning model to provide movie recommendations.
\item Worked with C++, Python.
\end{itemize}

% Tabulate Computer Skills; p{3in} defines paragraph 3 inches wide
\section{Computer \\ Skills}
   \begin{tabular}{l p{3in}}
    \underline{Languages:} & C++, Java, Python, HTML, Javascript \\

     \underline{Software:} & Flask server development, \LaTeX, vim
 \end{tabular}

\end{resume} 
\end{document} 




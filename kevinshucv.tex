\documentclass[10pt, letterpaper]{article}

% Packages:
\usepackage[
    ignoreheadfoot, % set margins without considering header and footer
    top=2 cm, % seperation between body and page edge from the top
    bottom=2 cm, % seperation between body and page edge from the bottom
    left=2 cm, % seperation between body and page edge from the left
    right=2 cm, % seperation between body and page edge from the right
    footskip=1.0 cm, % seperation between body and footer
    % showframe % for debugging 
]{geometry} % for adjusting page geometry
\usepackage{titlesec} % for customizing section titles
\usepackage{tabularx} % for making tables with fixed width columns
\usepackage{array} % tabularx requires this
\usepackage[dvipsnames]{xcolor} % for coloring text
\definecolor{primaryColor}{RGB}{0, 0, 0} % define primary color
\usepackage{enumitem} % for customizing lists
\usepackage{fontawesome5} % for using icons
\usepackage{amsmath} % for math
\usepackage[
    pdftitle={Kevin Shu's CV},
    pdfauthor={Kevin Shu},
    colorlinks=true,
    urlcolor=primaryColor
]{hyperref} % for links, metadata and bookmarks
\usepackage[pscoord]{eso-pic} % for floating text on the page
\usepackage{calc} % for calculating lengths
\usepackage{bookmark} % for bookmarks
\usepackage{lastpage} % for getting the total number of pages
\usepackage{changepage} % for one column entries (adjustwidth environment)
\usepackage{paracol} % for two and three column entries
\usepackage{ifthen} % for conditional statements
\usepackage{needspace} % for avoiding page brake right after the section title
\usepackage{iftex} % check if engine is pdflatex, xetex or luatex

% Ensure that generate pdf is machine readable/ATS parsable:
\ifPDFTeX
    \input{glyphtounicode}
    \pdfgentounicode=1
    \usepackage[T1]{fontenc}
    \usepackage[utf8]{inputenc}
    \usepackage{lmodern}
\fi

\usepackage{charter}

% Some settings:
\raggedright
\AtBeginEnvironment{adjustwidth}{\partopsep0pt} % remove space before adjustwidth environment
\pagestyle{empty} % no header or footer
\setcounter{secnumdepth}{0} % no section numbering
\setlength{\parindent}{0pt} % no indentation
\setlength{\topskip}{0pt} % no top skip
\setlength{\columnsep}{0.15cm} % set column seperation
\pagenumbering{gobble} % no page numbering

\titleformat{\section}{\needspace{4\baselineskip}\bfseries\large}{}{0pt}{}[\vspace{1pt}\titlerule]

\titlespacing{\section}{
    % left space:
    -1pt
}{
    % top space:
    0.3 cm
}{
    % bottom space:
    0.2 cm
} % section title spacing

\renewcommand\labelitemi{$\vcenter{\hbox{\small$\bullet$}}$} % custom bullet points
\newenvironment{highlights}{
    \begin{itemize}[
        topsep=0.10 cm,
        parsep=0.10 cm,
        partopsep=0pt,
        itemsep=0pt,
        leftmargin=0 cm + 10pt
    ]
}{
    \end{itemize}
} % new environment for highlights


\newenvironment{highlightsforbulletentries}{
    \begin{itemize}[
        topsep=0.10 cm,
        parsep=0.10 cm,
        partopsep=0pt,
        itemsep=0pt,
        leftmargin=10pt
    ]
}{
    \end{itemize}
} % new environment for highlights for bullet entries

\newenvironment{onecolentry}{
    \begin{adjustwidth}{
        0 cm + 0.00001 cm
    }{
        0 cm + 0.00001 cm
    }
}{
    \end{adjustwidth}
} % new environment for one column entries

\newenvironment{twocolentry}[2][]{
    \onecolentry
    \def\secondColumn{#2}
    \setcolumnwidth{\fill, 4.5 cm}
    \begin{paracol}{2}
}{
    \switchcolumn \raggedleft \secondColumn
    \end{paracol}
    \endonecolentry
} % new environment for two column entries

\newenvironment{threecolentry}[3][]{
    \onecolentry
    \def\thirdColumn{#3}
    \setcolumnwidth{, \fill, 4.5 cm}
    \begin{paracol}{3}
    {\raggedright #2} \switchcolumn
}{
    \switchcolumn \raggedleft \thirdColumn
    \end{paracol}
    \endonecolentry
} % new environment for three column entries

\newenvironment{header}{
    \setlength{\topsep}{0pt}\par\kern\topsep\centering\linespread{1.5}
}{
    \par\kern\topsep
} % new environment for the header

\newcommand{\placelastupdatedtext}{% \placetextbox{<horizontal pos>}{<vertical pos>}{<stuff>}
  \AddToShipoutPictureFG*{% Add <stuff> to current page foreground
    \put(
        \LenToUnit{\paperwidth-2 cm-0 cm+0.05cm},
        \LenToUnit{\paperheight-1.0 cm}
    ){\vtop{{\null}\makebox[0pt][c]{
        \small\color{gray}\textit{Last updated in September 2024}\hspace{\widthof{Last updated in September 2024}}
    }}}%
  }%
}%

% save the original href command in a new command:
\let\hrefWithoutArrow\href

% new command for external links:
\setlist[description]{font=\normalfont}


\begin{document}
    \newcommand{\AND}{\unskip
        \cleaders\copy\ANDbox\hskip\wd\ANDbox
        \ignorespaces
    }
    \newsavebox\ANDbox
    \sbox\ANDbox{$|$}

    \begin{header}
        \fontsize{25 pt}{25 pt}\selectfont Kevin Shu\\
        \vspace{5 pt}
        \normalsize
        \mbox{kshu@caltech.edu} \hspace{5 pt}%
        \mbox{\hrefWithoutArrow{https://kevinshu.me}{kevinshu.me}} \hspace{5 pt}
        \mbox{\today}
    \end{header}
\section{Work}
\begin{description}
    \item[2024-\qquad] {\bf Postdoctoral Researcher} (Computing and Mathematical Sciences),\\ {\hspace{0.35in} \bf California Institute of Technology}, Pasadena, CA
\end{description}
\section{Education} 
\begin{description}
    \item[2019-2024] {\bf PhD.,} Algorithms Combinatorics and Optimization,\\
        \hspace{0.35in} {\bf Georgia Institute of Technology,} Atlanta, GA
 
 \begin{itemize} \itemsep -2pt  % reduce space between items
 \item Advised by Grigoriy Blekherman in the Department of Mathematics.
 \end{itemize}

    \item[2014-2018] {\bf B.S.}, Mathematics, Computer Science,
\\ \hspace{0.35in} {\bf California Institute of Technology}, Pasadena, CA
\end{description}

As is traditional in pure mathematics, authors of publications are listed alphabetically, and author ordering does not reflect the contributions of any coauthors.
\section{Publications}
\begin{description}
    \item[\emph{Debiasing Polynomial and Fourier Regression}] (with Chris Camaño and Raphael A. Meyer), accepted at SOSA 2026.\\

    \item[\emph{Composing Optimized Stepsize Schedules for Gradient Descent}] (with Ben Grimmer and Alex L. Wang), to appear in Mathematics of Operations Research in 2025.\\

    \item[\emph{A Semidefinite Hierarchy for the Expected Independence Number of a Random Graph}] (with Diego Cifuentes, and Alejandro Toriello), Optimization Letters, 2025\\

    \item[\emph{Accelerated Objective Gap and Gradient Norm Convergence for Gradient Descent via Long Steps}] (with Ben Grimmer and Alex L. Wang), Informs Journal on Optimization, 2025\\

    \item[\emph{Symmetric Hyperbolic Polynomials}] (with Greg Blekherman and Julia Lindberg), Journal of Pure and Applied Algebra, 2025\\

    \item[\emph{Hidden convexity, optimization, and algorithms on rotation matrices  }] (with Akshay Ramachandran and Alex L. Wang), Mathematics of Operations Research, 2024\\

    \item[\emph{Linear Principal Minor Polynomials: Hyperbolic Determinantal Inequalities and Spectral Containment }] (with Greg Blekherman, Mario Kummer, Raman Sanyal and Shengding Sun), International Mathematics Research Notices, 2022\\

    \item[\emph{Hyperbolic Relaxation of k-Locally Positive Semidefinite Matrices}] (with Grigoriy Blekherman, Santanu Dey and Shengding Sun), SIAM Journal on Optimization, 2022\\

    \item[\emph{Sums of Squares and Sparse Semidefinite Programming}] (with Grigoriy Blekherman),
SIAM Journal for Applied Algebra and Geometry, 2021\\

    \item[\emph{Syntactic Structures and Code Parameters}] (with Matilde Marcolli), Mathematics in Computer Science, 2017
\end{description}

\section{Preprints}
\begin{description}
    \item[\emph{Lagrangian Dual Sections: A Topological Perspective on Hidden Convexity}] (with Venkat Chandrasekaran, Tim Duff, Jose Israel Rodriguez), \href{https://arxiv.org/abs/2510.06112}{arXiv:2510.06112}, in submission

    \item[\emph{Beyond Minimax Optimality: A Subgame Perfect Gradient Method}] (with Ben Grimmer and Alex Wang), \href{https://arxiv.org/abs/2412.06731}{arXiv:2412.06731}, in revision at Mathematical Programming

    \item[\emph{Accelerated Gradient Descent via Long Steps}] (with Ben Grimmer and Alex Wang), \href{https://arxiv.org/abs/2309.09961}{arXiv:2309.09961}, 2023

    \item[\emph{Quadratic Programming with Sparsity Constraints via Polynomial Roots}], 
\href{https://arxiv.org/abs/2208.11143}{arxiv:2208.11143}, 2022
\end{description}

\section{Talks and Presentations Given}
{\bf  Hidden Convexity,} presented at ICCOPT, 2025

{\bf  Hidden Convexity and the Rotation Group,} presented at the SIAM Conference on Applied Algebraic Geometry, 2025

{\bf  Algebraic Methods in Convex Optimization,} presented at the UCLA Math Colloquium, 2025

{\bf  Semialgebraic Methods in Convex Optimization,} presented at the Joint Mathematics Meetings, 2025 

{\bf  Hidden convexity, optimization, and algorithms on rotation matrices  ,} presented at the Informs Optimization Society conference, 2024 

{\bf  Symmetric Hyperbolic Polynomials,} presented at the SIAM Conference on Applied Algebraic Geometry, 2023 

{\bf  Sparse Regression and PCA via Polynomial Roots,} presented at the SIAM Conference on Optimization, 2023 

{\bf  Hyperbolicity Cones and Sparse Optimization,} presented at the MIT LIDS seminar, 2023 

{\bf  Symmetrically Hyperbolic Polynomials,} presented at the \"Oberwolfach Meeting on New Directions in Algebraic Geometry, 2023 

{\bf  Sparse Quadratic Programs via Polynomial Roots,} presented at the Carnegie Mellon University ACO seminar, 2023 

{\bf  Sparse Quadratic Programs via Polynomial Roots,} presented at the Centrum Wiskunde and Informatica Networks and Optimization seminar, 2022 

{\bf  Approximating Sparse Semidefinite Programs,} presented at the INFORMS conference, 2021 

{\bf  Poster on Sparse Semidefinite Programs,} presented at the MIP and IPCO conferences, 2021 

{\bf  Causal Inference and Optimization,} presented at the ACO Student Seminar, 2021

{\bf  Lightning Talk on Hyperbolic Relaxations of Locally-PSD Matrices ,} presented at the ICERM - Symmetry, Randomness, and Computations in Real Algebraic Geometry.
, 2020



\section{Academic Honors} 
2025 Best Thesis Award for the Georgia Tech Mathematics Department\\
2022 ACO-ARC Fellowship\\
2022 ARCS Foundation award\\
2021 Honorable Mention at the MIP Conference Poster Competition\\
2021 Honorable Mention at the IPCO Conference Poster Competition\\
2021 David L. Brown Fellowship from the Georgia Tech Math Department\\
2018 National Science Foundation Graduate Research Fellowship Recipient\\
2018 Georgia Institute of Technology President's Fellowship Recipient

\section{Conference Organization}
{\bf Coorganizer for the Session on Algebraic Methods in Optimization,} ICCOPT    \\       July 2025

{\bf Coorganizer for the Georgia Tech Student Algebra Seminar,} Georgia Tech, GA    \\       August 2022-December 2023

{\bf Coorganizer for the Special Session on Convexity,} SIAM Conference on Applied Algebraic Geometry, Georgia Tech, GA    \\       July 2023

{\bf Organizer for the AMS Special Session on Algebraic Methods in Algorithms,} Spring 2023 Southeastern Section Meeting of the AMS, Georgia Tech, GA    \\       March 2023

\section{Research Experience}
{\bf Visiting Scholar,} Max-Planck Institute for Mathematics in the Sciences, Leipzig, Germany \\ Summer 2022
\begin{itemize} \itemsep -2pt %reduce space between items
\item Working under the supervision of Rainer Sinn and Bernd Sturmfels.
\end{itemize}
{\bf Research Assistantship,} Georgia Tech, Atlanta, GA \\ Summer 2020
\begin{itemize} \itemsep -2pt %reduce space between items
\item Funded in part by NSF grant DMS-1901950 and the ACO department.
\item Advised by Grigoriy Blekherman.
\end{itemize}

\section{Outreach and Community Service} 
       {\bf Representative for the Diversity, Equity, and Inclusion committee,} Georgia Tech, GA    \\         2022-2023

       {\bf First Year Mentor,} Georgia Tech, Atlanta, Georgia     \\         2020-2021 

       {\bf Directed Reading Program Mentor,} Georgia Tech, Atlanta, Georgia     \\         2020-2021 

       {\bf Senior Class President,} Caltech, Pasadena, CA    \\         2018-2019 

		{\bf Board of Control Secretary,} Caltech \\   2017

\section{Teaching Experience}
{\bf Differential Equations Teaching Assistant,} Georgia Tech, Atlanta, GA \\ Aug 2022-Dec 2022

{\bf Differential Equations Teaching Assistant,} Georgia Tech, Atlanta, GA \\ Aug 2021-Dec 2021

{\bf Number Theory Lecture Assistant,} Georgia Tech, Atlanta, GA \\ Jan 2021-May 2021

{\bf Differential Equations Teaching Assistant,} Georgia Tech, Atlanta, GA \\ Jan 2020-May 2020

{\bf Linear Algebra Teaching Assistant,} Georgia Tech, Atlanta, GA \\ Aug 2019-Dec 2019

{\bf Advanced Algorithms Teaching Assistant,} Caltech, Pasadena, CA \\ Jan 2018-Mar 2018

{\bf Linear Algebra Teaching Assistant,} Caltech, Pasadena, CA \\ Sep 2017-Dec 2017

{\bf Introduction to Algorithms Teaching Assistant,} Caltech, Pasadena, CA \\ Jan 2017-Mar 2017

 \section{Work\\ Experience}
 {\bf Full-time Software Engineer,} Google, Mountain View, CA \\ August 2018-July 2019
 \begin{itemize} \itemsep -2pt  % reduce space between items
 \item Full stack web development for a data labelling service (Crowd-Compute)
 \item Lead an initiative to update authentication/authorization to more modern technologies.
 \item Added a major feature for tracking work in the system.
 \item Managed production releases and infrastructure issues.
 \item Worked with C++, Java.
 \end{itemize}
 
{\bf Software Engineering Intern,} Google, Mountain View, CA \\ Aug 2018-Jul 2019
\begin{itemize} \itemsep -2pt %reduce space between items
\item Gathered data from online sources by parsing Reddit pages.
\item Built a machine learning model to provide movie recommendations.
\item Worked with C++, Python.
\end{itemize}

\end{document}
